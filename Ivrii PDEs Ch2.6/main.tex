\documentclass{article}
\title{Intro to PDEs: Ch 2.6 HW}
\author{Logan Rhyne, Harley Combest, Roy Galang, Jesse DiCenso}
\usepackage[T1]{fontenc}
\usepackage{amsfonts, amsmath, amsthm, amssymb}
\usepackage{mathtools, bigints, empheq}
\usepackage{graphicx, wrapfig, xcolor, float}
\usepackage{stackrel}
\usepackage{pgfplots}
\usepackage[shortlabels]{enumitem}
\usepackage[margin=1.0in]{geometry}
\setlength{\parindent}{0pt}
\theoremstyle{definition}
\newtheorem*{lemma}{Lemma}
\newtheorem*{conj}{Conjecture}
\newtheorem{prob}{}
\newtheorem*{pf}{Proof}
\newtheorem*{dpf}{Disproof}
\renewcommand\qedsymbol{$\blacksquare$}
\renewcommand{\emptyset}{\varnothing}
\renewcommand{\epsilon}{\varepsilon}
\newenvironment{disproof}{\begin{proof}[Disproof]}{\end{proof}}
\newenvironment{ans}{\begin{proof}[Answer]\renewcommand{\qedsymbol}{}}{\end{proof}}
\newenvironment{boldenv}{\bfseries\boldmath}{}
\newcommand{\N}{\mathbb{N}}
\newcommand{\Z}{\mathbb{Z}}
\newcommand{\R}{\mathbb{R}}

\pgfplotsset{compat=1.18}

\DeclareMathOperator{\ran}{range}

\begin{document}
	
    \maketitle

\begin{boldenv}
    \underline{Problem 4}. \begin{enumerate}
        \item Find solution
        \[\begin{cases}
            u_{tt} - c^2u_{xx} = 0 & t > 0, x > 0\\
            u|_{t=0} = \phi(x),\quad u_t|_{t=0} = c\phi'(x) & x > 0\\
            (u_x + \alpha u)|_{x=0} = 0 & t > 0
        \end{cases}\]
        (separately in $x > ct$ and $0 < x < ct$).

        \item Discuss \textit{reflected wave}. In particular, consider $\phi(x) = e^{ikx}$.
    \end{enumerate}
\end{boldenv}
\begin{ans}
\begin{enumerate}
    \item For $x > ct$, since the boundary has not come into effect yet, we can just use D'Alembert's formula. From this, we get
    \begin{align*}
        u(t,x)|_{x > ct} &= \frac{1}{2}(\phi(x+ct) + \phi(x-ct)) + \frac{1}{2c}\int_{x-ct}^{x+ct}c\phi'(s)\,ds\\
        &= \frac{1}{2}(\phi(x+ct) + \phi(x-ct)) + \frac{1}{2}(\phi(x+ct) - \phi(x-ct)) + C\\
        &= \phi(x+ct) + C
    \end{align*}

    Now, for $0 < x < ct$, we utilize the fact that we are working with a pure reflection to get a general solution that
    \[u(t,x) = \phi(x+ct) + g(x-ct).\]
    We plug this into the boundary condition to get that
    \[\phi'(ct) + g'(-ct) + \alpha \phi(ct) + \alpha g(-ct) = 0.\]
    Now we want to solve for $g$, and we substitute $s = x-ct$, getting
    \begin{align*}
        g'(s) + \alpha g(s) &= -(\phi'(-s) + \alpha \phi(-s))\\
        e^{\alpha s}(g'(s) + \alpha g(s)) &= -e^{\alpha s}(\phi'(-s) + \alpha \phi(-s))\\
        (e^{\alpha s}g(s))' &= -e^{\alpha s}(\phi'(-s) + \alpha \phi(-s))\\
        e^{\alpha s}g(s) &= \int_0^{-s} e^{-\alpha\tau}(\phi'(\tau) + \alpha\, \phi(\tau))\,d\tau \tag{taking $\tau = -s$}\\
        g(s) &= e^{-\alpha s}\int_0^{-s} e^{-\alpha\tau}(\phi'(\tau) + \alpha\, \phi(\tau))\,d\tau + C\\
    \end{align*}

    To fit the initial conditions, we find that our constants are equal to zero. We now have our final solution that
    \[\boxed{u(t,x) = \begin{cases}
        \phi(x+ct) & x > ct\\
        \phi(x+ct) + e^{\alpha (ct-x)}\int_0^{ct-x} e^{-\alpha\tau}(\phi'(\tau) + \alpha\, \phi(\tau))\,d\tau & x < ct
    \end{cases}}\]

    \item The reflected wave in particular is the term of 
    \begin{equation}
        e^{\alpha (ct-x)}\int_0^{ct-x} e^{-\alpha\tau}(\phi'(\tau) + \alpha\, \phi(\tau))\,d\tau.
    \end{equation}
    We take $\phi(x) = e^{ikx}$, which means $\phi'(x) = ike^{ikx}$. Substituting this into expression (1) gives us
    \begin{equation*}
        e^{\alpha (ct-x)}\int_0^{ct-x} e^{-\alpha\tau}(ike^{ik\tau} + \alpha e^{ik\tau})\,d\tau.
    \end{equation*}
    We evaluate this integral, giving us
    \begin{align*}
        e^{\alpha (ct-x)}\int_0^{ct-x} e^{-\alpha\tau}(ike^{ik\tau} + \alpha e^{ik\tau})\,d\tau &= (ik + \alpha) e^{\alpha (ct-x)}\int_0^{ct-x}e^{\tau(ik - \alpha)}d\tau\\
        &= \frac{ik + \alpha}{ik - \alpha} e^{\alpha (ct-x)} \left[e^{\tau(ik - \alpha)}\right]_{\tau = 0}^{\tau = ct - x}\\
        &= \frac{ik + \alpha}{ik - \alpha}\left[ e^{\alpha (ct-x)} e^{(ct-x)(ik-\alpha)} -  e^{\alpha (ct-x)}\right]\\
        &= \frac{ik + \alpha}{ik - \alpha}\left[ e^{ik(ct-x)} - e^{\alpha(ct-x)} \right] \tag{2}
    \end{align*}

    It hits the boundary at $x=0$, so we plug that into (2) to get
    \begin{equation*}
        \frac{ik + \alpha}{ik - \alpha}e^{ikct} - \frac{ik + \alpha}{ik - \alpha}e^{\alpha ct}. \tag{3}
    \end{equation*}

    We mainly care about the second term of (3). If $\alpha > 0$, then the boundary puts energy into the wave. If $\alpha < 0$, the boundary takes energy out of the wave. If $\alpha = 0$, then the boundary does not affect the energy of the wave.
    
\end{enumerate}
\end{ans}

\begin{boldenv}
    \underline{Problem 5}. \begin{enumerate}
        \item Find solution
        \[\begin{cases}
            u_{tt} - c^2u_{xx} = 0 & t > 0, x > 0\\
            u|_{t=0} = \phi(x),\quad u_t|_{t=0} = c\phi'(x) & x > 0\\
            (u_x + \alpha u_t)|_{x=0} = 0 & t > 0
        \end{cases}\]
        (separately in $x > ct$ and $0 < x < ct$).

        \item Discuss \textit{reflected wave}. In particular, consider $\phi(x) = e^{ikx}$.
    \end{enumerate}
\end{boldenv}
\begin{ans}
\begin{enumerate}
    \item For $x > ct$, since the boundary has not come into effect yet, we can just use D'Alembert's formula. From this, we get
    \begin{align*}
        u(t,x)|_{x > ct} &= \frac{1}{2}(\phi(x+ct) + \phi(x-ct)) + \frac{1}{2c}\int_{x-ct}^{x+ct}c\phi'(s)\,ds\\
        &= \frac{1}{2}(\phi(x+ct) + \phi(x-ct)) + \frac{1}{2}(\phi(x+ct) - \phi(x-ct)) + C\\
        &= \phi(x+ct) + C
    \end{align*}

    Now, for $0 < x < ct$, we utilize the fact that we are working with a pure reflection to get a general solution that
    \[u(t,x) = \phi(x+ct) + g(x-ct).\]
    We plug this into the boundary condition to get that
    \begin{align*}
        \phi'(ct) + g'(-ct) + c\alpha \phi'(ct) + -c\alpha g'(-ct) &= 0\\
        \phi'(ct)(1 + c\alpha) + g'(-ct)(1 - c\alpha) &= 0
    \end{align*}
    Now we want to solve for $g$, and we substitute $s = x-ct$, getting
    \begin{align*}
        g'(s)(1-c\alpha) &= -\phi'(-s)(1+c\alpha)\\
        g'(s) &= -\frac{1+c\alpha}{1-c\alpha}\phi'(-
        s)\\
        g(s) &= \frac{1+c\alpha}{1-c\alpha}\int_0^{-s}\phi'(\tau)\,d\tau \tag{taking $\tau = -s$}\\
        &=\frac{1+c\alpha}{1-c\alpha}(\phi'(-s) - \phi'(0))\\
        &=\frac{1+c\alpha}{1-c\alpha}(\phi'(ct-x) - \phi'(0))
    \end{align*}

    To fit the initial conditions, we find that our constants are equal to zero. We now have our final solution that
    \[\boxed{u(t,x) = \begin{cases}
        \phi(x+ct) & x > ct\\
        \phi(x+ct) + \frac{1+c\alpha}{1-c\alpha}(\phi'(ct-x) - \phi'(0)) & x < ct
    \end{cases}}\]


\item The reflected wave in particular is the term of 
    \begin{equation*}
        \frac{1+c\alpha}{1-c\alpha}(\phi'(ct-x) - \phi'(0))
    \end{equation*}
    We take $\phi(x) = e^{ikx}$, which means $\phi'(x) = ike^{ikx}$. Substituting this into expression (1) gives us
    \begin{equation*}
        \frac{1+c\alpha}{1-c\alpha}(ike^{ik(ct-x)} - ik) \tag{4}
    \end{equation*}
    It hits the boundary at $x=0$, so we plug that into (4) to get
    \begin{equation*}
        ik\frac{1+c\alpha}{1-c\alpha}(e^{ikct} - 1)
    \end{equation*}

    There are no pure exponentials that rely on $\alpha$, so the boundary will not increase or decrease the amplitude of the wave depending on $\alpha$, but it will increase the frequency with time due to $e^{ikct}$.
\end{enumerate}
\end{ans}

\begin{boldenv}
    \underline{Problem 8}. \\
    By the method of continuation combined with D'Alembert's formula, solve each of the following twelve problems:
    \begin{enumerate}[(1), start=9]
        \item \[\begin{cases}
            u_{tt} - c^2u_{xx} = 0 & x > 0\\
            u|_{t=0} = 0 & x > 0\\
            u_t|_{t=0} = \cos{x} & x > 0\\
            u|_{x = 0} = 0 & t > 0
        \end{cases}\]
        \item \[\begin{cases}
            u_{tt} - c^2u_{xx} = 0 & x > 0\\
            u|_{t=0} = 0 & x > 0\\
            u_t|_{t=0} = \cos{x} & x > 0\\
            u_x|_{x = 0} = 0 & t > 0
        \end{cases}\]
        \item \[\begin{cases}
            u_{tt} - c^2u_{xx} = 0 & x > 0\\
            u|_{t=0} = 0 & x > 0\\
            u_t|_{t=0} = \sin{x} & x > 0\\
            u|_{x = 0} = 0 & t > 0
        \end{cases}\]
        \item \[\begin{cases}
            u_{tt} - c^2u_{xx} = 0 & x > 0\\
            u|_{t=0} = 0 & x > 0\\
            u_t|_{t=0} = \sin{x} & x > 0\\
            u_x|_{x = 0} = 0 & t > 0
        \end{cases}\]
        \item \[\begin{cases}
            u_{tt} - c^2u_{xx} = 0 & x > 0\\
            u|_{t=0} = \cos{x} & x > 0\\
            u_t|_{t=0} = 0 & x > 0\\
            u|_{x = 0} = 0 & t > 0
        \end{cases}\]
        \item \[\begin{cases}
            u_{tt} - c^2u_{xx} = 0 & x > 0\\
            u|_{t=0} = \cos{x} & x > 0\\
            u_t|_{t=0} = 0 & x > 0\\
            u_x|_{x = 0} = 0 & t > 0
        \end{cases}\]
        \item \[\begin{cases}
            u_{tt} - c^2u_{xx} = 0 & x > 0\\
            u|_{t=0} = \sin{x} & x > 0\\
            u_t|_{t=0} = 0 & x > 0\\
            u|_{x = 0} = 0 & t > 0
        \end{cases}\]
        \item \[\begin{cases}
            u_{tt} - c^2u_{xx} = 0 & x > 0\\
            u|_{t=0} = \sin{x} & x > 0\\
            u_t|_{t=0} = 0 & x > 0\\
            u_x|_{x = 0} = 0 & t > 0
        \end{cases}\]
        \item \[\begin{cases}
            u_{tt} - c^2u_{xx} = 0 & x > 0\\
            u|_{t=0} = 1 & x > 0\\
            u_t|_{t=0} = 0 & x > 0\\
            u_x|_{x = 0} = 0 & t > 0
        \end{cases}\]
        \item \[\begin{cases}
            u_{tt} - c^2u_{xx} = 0 & x > 0\\
            u|_{t=0} = 1 & x > 0\\
            u_t|_{t=0} = 0 & x > 0\\
            u|_{x = 0} = 0 & t > 0
        \end{cases}\]
        \item \[\begin{cases}
            u_{tt} - c^2u_{xx} = 0 & x > 0\\
            u|_{t=0} = 0 & x > 0\\
            u_t|_{t=0} = 1 & x > 0\\
            u_x|_{x = 0} = 0 & t > 0
        \end{cases}\]
        \item \[\begin{cases}
            u_{tt} - c^2u_{xx} = 0 & x > 0\\
            u|_{t=0} = 0 & x > 0\\
            u_t|_{t=0} = 1 & x > 0\\
            u|_{x = 0} = 0 & t > 0
        \end{cases}\]
    \end{enumerate}
\end{boldenv}
\begin{ans}
\begin{enumerate}[(1), start=9]
    \item In our problem, no waves are generated on the boundary. As such, we can solve the problem like a full-line problem via an odd extension (since $u|_{x=0} = 0$):
    \[\phi_{odd}(x) = \begin{cases}
        \phi(x) & x > 0\\
        0 & x = 0\\
        -\phi(-x) & x < 0
    \end{cases}.\]
    This allows us to rewrite the problem using D'Alembert's formula as
    \[u(x,t) = \frac{1}{2}(\phi_{odd}(x+ct) + \phi_{odd}(x-ct)) + \frac{1}{2c}\int_{x-ct}^{x+ct}\psi_{odd}(s)\,ds.\]
    There are two cases to consider: $x \geq ct$ and $0 < x < ct$.\\
    \underline{$x \geq ct$}: By the odd extension, this instance yields D'Alembert's formula
    \[u(x,t) = \frac{1}{2}(\phi(x+ct) + \phi(x-ct)) + \frac{1}{2c}\int_{x-ct}^{x+ct}\psi(s)\,ds.\]
    \underline{$0 < x < ct$}: The odd extension says to change the sign of both $\phi$ and $x$. As such, we have
    \[u(x,t) = \frac{1}{2}(\phi(x+ct) - \phi(ct-x)) + \frac{1}{2c}\int_{ct-x}^{x+ct}\psi(s)\,ds\]
    For the purposes of this problem, we take $\phi(x) = 0$ and $\psi(x) = \cos(x)$. We integrate in the first case to get 
    \begin{align*}
        \frac{1}{2c}\int_{x-ct}^{x+ct}\cos(s)ds &= \left.\frac{1}{2c}\sin(s)\right|_{x-ct}^{x+ct}\\
        &=\frac{1}{2c}(\sin(x+ct) - \sin(x-ct))
    \end{align*}

    We integrate in the second case to get
    \begin{align*}
        \frac{1}{2c}\int_{ct-x}^{x+ct}\cos(s)ds &= \left.\frac{1}{2c}\sin(s)\right|_{ct-x}^{x+ct}\\
        &= \frac{1}{2c}(\sin(x+ct) - \sin(ct-x))
    \end{align*}
    We now have our final solution that
    \[\boxed{u(x,t) = \begin{cases}
       \frac{1}{2c}(\sin(x+ct) - \sin(x-ct)) & x \geq ct\\
        \frac{1}{2c}(\sin(x+ct) - \sin(ct-x)) & 0 \leq x < ct
    \end{cases}}\]

    \item In our problem, no waves are generated on the boundary. As such, we can solve the problem like a full-line problem via an even extension (since $u_x|_{x=0} = 0$):
    \[\phi_{even}(x) = \begin{cases}
        \phi(x) & x > 0\\
        0 & x = 0\\
        \phi(-x) & x < 0
    \end{cases}.\]
    This allows us to rewrite the problem using D'Alembert's formula as
    \[u(x,t) = \frac{1}{2}(\phi_{even}(x+ct) + \phi_{even}(x-ct)) + \frac{1}{2c}\int_{x-ct}^{x+ct}\psi_{even}(s)\,ds.\]
    There are two cases to consider: $x \geq ct$ and $0 < x < ct$.\\
    \underline{$x \geq ct$}: By the even extension, this instance yields D'Alembert's formula
    \[u(x,t) = \frac{1}{2}(\phi(x+ct) + \phi(x-ct)) + \frac{1}{2c}\int_{x-ct}^{x+ct}\psi(s)\,ds.\]
    \underline{$0 \leq x < ct$}: Otherwise, we have to modify it per the definition of the even extension. As such, we have
    \[u(x,t) = \frac{1}{2}(\phi(x+ct) + \phi(ct-x)) + \frac{1}{2c}\left[\int_0^{x+ct}\psi(s)\,ds + \int_0^{ct-x}\psi(s)\,ds\right]\]
    For the purposes of this problem, we take $\phi(x) = 0$ and $\psi(x) = \cos(x)$. We integrate in the first case to get 
    \begin{align*}
        \frac{1}{2c}\int_{x-ct}^{x+ct}\cos(s)ds &= \left.\frac{1}{2c}\sin(s)\right|_{x-ct}^{x+ct}\\
        &=\frac{1}{2c}(\sin(x+ct) - \sin(x-ct))
    \end{align*}

    We integrate in the second case to get
    \begin{align*}
        \frac{1}{2c}\left[\int_0^{x+ct}\cos(s)\,ds + \int_0^{ct-x}\cos(s)\,ds\right]&= \left.\frac{1}{2c}\sin(s)\right|_0^{x+ct} + \left.\frac{1}{2c}\sin(s)\right|_0^{ct-x}\\
        &= \frac{1}{2c}(\sin(x+ct) + \sin(ct-x))
    \end{align*}
    We now have our final solution that
    \[\boxed{u(x,t) = \begin{cases}
       \frac{1}{2c}(\sin(x+ct) - \sin(x-ct)) & x \geq ct\\
        \frac{1}{2c}(\sin(x+ct) + \sin(ct-x)) & 0 \leq x < ct
    \end{cases}}\]

    \item In our problem, no waves are generated on the boundary. As such, we can solve the problem like a full-line problem via an odd extension (since $u|_{x=0} = 0$):
    \[\phi_{odd}(x) = \begin{cases}
        \phi(x) & x > 0\\
        0 & x = 0\\
        -\phi(-x) & x < 0
    \end{cases}.\]
    This allows us to rewrite the problem using D'Alembert's formula as
    \[u(x,t) = \frac{1}{2}(\phi_{odd}(x+ct) + \phi_{odd}(x-ct)) + \frac{1}{2c}\int_{x-ct}^{x+ct}\psi_{odd}(s)\,ds.\]
    There are two cases to consider: $x \geq ct$ and $0 < x < ct$.\\
    \underline{$x \geq ct$}: By the odd extension, this instance yields D'Alembert's formula
    \begin{align*}
        u(x,t) &= \frac{1}{2}(\phi(x+ct) + \phi(x-ct)) + \frac{1}{2c}\int_{x-ct}^{x+ct}\psi(s)\,ds\\
        u(x,t) &= \frac{1}{2c}\int_{x-ct}^{x+ct}\sin{(s)}\,ds
    \end{align*}
    \underline{$0 < x < ct$}: The odd extension says to change the sign of both $\phi$ and $x$. As such, we have
    \[u(x,t) = \frac{1}{2}(\phi(x+ct) - \phi(ct-x)) + \frac{1}{2c}\int_{ct-x}^{x+ct}\psi(s)\,ds\]
    For the purposes of this problem, we take $\phi(x) = 0$ and $\psi(x) = \sin(x)$. We integrate in the first case to get 
    \begin{align*}
        \frac{1}{2c}\int_{x-ct}^{x+ct}\sin(s)ds &= -\left.\frac{1}{2c}\cos(s)\right|_{x-ct}^{x+ct}\\
        &=\frac{1}{2c}(\cos(x-ct) - \cos(x+ct))
    \end{align*}

    We integrate in the second case to get
    \begin{align*}
        \frac{1}{2c}\int_{ct-x}^{x+ct}\sin(s)ds &= -\left.\frac{1}{2c}\cos(s)\right|_{ct-x}^{x+ct}\\
        &= -\frac{1}{2c} [\cos{(x+ct)} - \cos{(ct-x)}]
    \end{align*}
    Since cosine is even, $\cos(-x) = \cos(x)$, so we now have our final solution that, when $0 < x < ct$,
    \[u(x,t) = \frac{1}{2c}(\cos(x-ct) - \cos(x+ct))\]
    Therefore, our final solution is
    \[\boxed{u(x,t) = \begin{cases}
        \frac{1}{2c}(\cos{(x-ct)} - \cos{(x+ct)} ) & x \geq ct\\
        \frac{1}{2c}(\cos{(x-ct)} - \cos{(x+ct)} ) & 0 \leq x < ct
    \end{cases}}\]

    \item In this problem, $g(x)=0 \text{ and } h(x) = \sin{(x)}$ and there is a Neumann condition, meaning that this problem can be solved with an even extension. Therefore, we have
    \[\phi_{even}(x) = \begin{cases}
        \phi(x) & x > 0\\
        0 & x = 0\\
        \phi(-x) & x < 0
    \end{cases}.\]
    and D'Alembert's formula can be rewritten as
    \[u(x,t) = \frac{1}{2}(\phi_{even}(x+ct) + \phi_{even}(x-ct)) + \frac{1}{2c}\int_{x-ct}^{x+ct}\psi_{even}(s)\,ds.\]
    \underline{$x \geq ct$}: This is just the normal form of D'Alembert's formula:
    \begin{align*}
        u(x,t) &= \frac{1}{2c}\int_{x-ct}^{x+ct}\psi_{even}(s)\,ds\\
        u(x,t) &= \frac{1}{2c}\int_{x-ct}^{x+ct}\sin{(s)}\,ds\\
        u(x,t) &= -\frac{1}{2c} \cos{(s)}|_{x-ct}^{x+ct}\,ds\\
        u(x,t) &= \frac{1}{2c} [\cos{(x-ct)} - \cos{(x+ct)}]
    \end{align*}
    \underline{$0 < x < ct$}: The even function will affect D'Alembert's formula and will now be written as:
    \[u(x,t) = \frac{1}{2}(g(x+ct) + g(ct-x)) + \frac{1}{2c}\left[\int_0^{x+ct}h(s)\,ds + \int_0^{ct-x}h(s)\,ds\right]\]
    Since $g(x) = 0$, we only have to worry about the integration:
    \begin{align*}
        u(x,t) &= \frac{1}{2c} \left[\int_0^{x+ct}h(s)\,ds + \int_0^{ct-x}h(s)\,ds\right]\\
        u(x,t) &= \frac{1}{2c} \left[\int_0^{x+ct}\sin{(s)}\,ds + \int_0^{ct-x}\sin{(s)}\,ds\right]\\
        u(x,t) &= \frac{1}{2c} \left[ -\cos{(s)}|_0^{x+ct} - \cos{(s)}|_0^{ct-x} \right]\\
        u(x,t) &= \frac{1}{2c} \left[ 1 - \cos{(x+ct)} - [ \cos{(ct-x)} - 1 ] \right]\\
        u(x,t) &= \frac{1}{2c} \left[ 2 - \cos{(x+ct)} - \cos{(ct-x)} \right]\\
    \end{align*}
    The solution for this wave equation is
    \[\boxed{u(x,t) = \begin{cases}
        \frac{1}{2c}(\cos{(x-ct)} - \cos{(x+ct)} ) & x \geq ct\\
        \frac{1}{2c}(2 - \cos{(x+ct)} - \cos{(ct-x)} ) & 0 \leq x < ct
    \end{cases}}\]

    \item Since no waves are generated on the boundary, we can solve this like a full line problem via an odd extension. Define
    \[\phi_{odd}(x) = \begin{cases}
        \phi(x) & x > 0\\
        0 & x = 0\\
        -\phi(-x) & x < 0
    \end{cases}.\]
    We can now use D'Alembert's Formula
    \[u(x,t) = \frac{1}{2}(\phi_{odd}(x+ct) + \phi_{odd}(x-ct)) + \frac{1}{2c}\int_{x-ct}^{x+ct}\psi_{odd}(s)\,ds.\]
    For $x \geq ct$, this results in the regular form of D'Alembert's formula:
    \[u(x,t) = \frac{1}{2}(\phi(x+ct) + \phi(x-ct)) + \frac{1}{2c}\int_{x-ct}^{x+ct}\psi(s)\,ds.\]
    Otherwise, we have to modify it per the definition of the odd extension:
    \[u(x,t) = \frac{1}{2}(\phi(x+ct) - \phi(ct-x)) + \frac{1}{2c}\int_{ct-x}^{x+ct}\psi(s)\,ds\]
    For the purposes of this problem, we take $\phi(x) = \cos(x)$ and $\psi(x) = 0$. Note that since cosine is an even function, $\cos(x) = \cos(-x)$. We now have our final solution that
    \[\boxed{u(x,t) = \begin{cases}
        \frac{1}{2}(\cos(x+ct) + \cos(x-ct)) & x \geq ct\\
        \frac{1}{2}(\cos(x+ct) - \cos(x-ct)) & 0 \leq x < ct
    \end{cases}}\]

    \item This problem has a Neumann condition, so we have to utilize an even extension.
    \[\phi_{even}(x) = \begin{cases}
        \phi(x) & x > 0\\
        0 & x = 0\\
        \phi(-x) & x < 0
    \end{cases}.\]
    Which rewrites D'Alembert's equation as
    \[u(x,t) = \frac{1}{2}(\phi_{even}(x+ct) + \phi_{even}(x-ct)) + \frac{1}{2c}\int_{x-ct}^{x+ct}\psi_{even}(s)\,ds.\]
    Since $\psi{(x)} = 0$, no integration is needed, and we only use the first part of the equation is used.
    \underline{$x \geq ct$}: The equation just becomes D'Alembert's equation.
    \begin{align*}
        u(x,t) &= \frac{1}{2}(\phi_{even}(x+ct) + \phi_{even}(x-ct))\\
        u(x,t) &= \frac{1}{2}(\cos{(x+ct)} + \cos{(x-ct)})
    \end{align*}
    \underline{$0 < x < ct$}: We have to use a rewritten form of the equation for these:
    \begin{align*}
        u(x,t) &= \frac{1}{2}(\phi(x+ct) + \phi(ct-x)) + \frac{1}{2c}\left[\int_0^{x+ct}\psi(s)\,ds + \int_0^{ct-x}\psi(s)\,ds\right]\\
        u(x,t) &= \frac{1}{2}(\cos{(x+ct)} + \cos{(ct-x)})
    \end{align*}
    Therefore, the solution to the wave equation with these boundary conditions is:
    \[\boxed{u(x,t) = \begin{cases}
        \frac{1}{2}(\cos{(x+ct)} + \cos{(x-ct)}) & x \geq ct\\
        \frac{1}{2}(\cos{(x+ct)} + \cos{(ct-x)}) & 0 \leq x < ct
    \end{cases}}\]

    \item Since no waves are generated on the boundary, we can solve this like a full line problem via an odd extension. Define
    \[\phi_{odd}(x) = \begin{cases}
        \phi(x) & x > 0\\
        0 & x = 0\\
        -\phi(-x) & x < 0
    \end{cases}.\]
    We can now use D'Alembert's Formula
    \[u(x,t) = \frac{1}{2}(\phi_{odd}(x+ct) + \phi_{odd}(x-ct)) + \frac{1}{2c}\int_{x-ct}^{x+ct}\psi_{odd}(s)\,ds.\]
    For $x > ct$, this results in the regular form of D'Alembert's formula:
    \[u(x,t) = \frac{1}{2}(\phi(x+ct) + \phi(x-ct)) + \frac{1}{2c}\int_{x-ct}^{x+ct}\psi(s)\,ds.\]
    Otherwise, we have to modify it per the definition of the odd extension:
    \[u(x,t) = \frac{1}{2}(\phi(x+ct) - \phi(ct-x)) + \frac{1}{2c}\int_{ct-x}^{x+ct}\psi(s)\,ds\]
    For the purposes of this problem, we take $\phi(x) = \sin(x)$ and $\psi(x) = 0$. Note that since sine is an odd function, $\sin(x) = -\sin(-x)$. We now have our final solution that
    \[\boxed{u(x,t) = \frac{1}{2}(\sin(x+ct) + \sin(x-ct))}\]

    \item As the initial boundary conditions show, we can utilize an even extension to solve this:
    \[\phi_{even}(x) = \begin{cases}
        \phi(x) & x > 0\\
        0 & x = 0\\
        \phi(-x) & x < 0
    \end{cases}.\]
    D'Alembert's equation now looks like this:
    \[u(x,t) = \frac{1}{2}(\phi_{even}(x+ct) + \phi_{even}(x-ct)) + \frac{1}{2c}\int_{x-ct}^{x+ct}\psi_{even}(s)\,ds.\]
    $\psi{(x)} = 0$, so integration is not needed.\\
    \underline{$x \geq ct$}: This is just D'Alembert's equation.
    \begin{align*}
        u(x,t) &= \frac{1}{2}(\phi_{even}(x+ct) + \phi_{even}(x-ct))\\
        u(x,t) &= \frac{1}{2}(\sin{(x+ct)} + \sin{(x-ct)})
    \end{align*}
    \underline{$0 < x < ct$}: The rewritten form of D'Alembert's equation now looks like this:
    \begin{align*}
        u(x,t) &= \frac{1}{2}(\phi(x+ct) + \phi(ct-x)) + \frac{1}{2c}\left[\int_0^{x+ct}\psi(s)\,ds + \int_0^{ct-x}\psi(s)\,ds\right]\\
        u(x,t) &= \frac{1}{2}(\sin{(x+ct)} + \sin{(ct-x)})
    \end{align*}
    Therefore, the solution to the wave equation with these boundary conditions is:
    \[\boxed{u(x,t) = \begin{cases}
        \frac{1}{2}(\sin{(x+ct)} + \sin{(x-ct)}) & x \geq ct\\
        \frac{1}{2}(\sin{(x+ct)} + \sin{(ct-x)}) & 0 \leq x < ct
    \end{cases}}\]

    \item Since no waves are generated on the boundary, we can solve this like a full line problem via an even extension (Neumann boundary conditions). Define
    \[\phi_{even}(x) = \begin{cases}
        \phi(x) & x > 0\\
        0 & x = 0\\
        \phi(-x) & x < 0
    \end{cases}.\]
    We can now use D'Alembert's Formula
    \[u(x,t) = \frac{1}{2}(\phi_{even}(x+ct) + \phi_{even}(x-ct)) + \frac{1}{2c}\int_{x-ct}^{x+ct}\psi_{even}(s)\,ds.\]
    For $x > ct$, this results in the regular form of D'Alembert's formula:
    \[u(x,t) = \frac{1}{2}(\phi(x+ct) + \phi(x-ct)) + \frac{1}{2c}\int_{x-ct}^{x+ct}\psi(s)\,ds.\]
    Otherwise, we have to modify it per the definition of the even extension:
    \[u(x,t) = \frac{1}{2}(\phi(x+ct) + \phi(ct-x)) + \frac{1}{2c}\left(\int_{0}^{ct-x}\psi(s)\,ds+\int_{0}^{x+ct} \psi(s)\,ds\right)\]
    For the purposes of this problem, we take $\phi(x) = 1$ and $\psi(x) = 0$. We now have our final solution that
    \[\boxed{u(x,t) = \begin{cases}
        1 & x \geq ct\\
        1 & 0 \leq x < ct
    \end{cases}}\]

    \item Since no waves are generated on the boundary, we can solve this like a full line problem via an odd extension. Define
    \[\phi_{odd}(x) = \begin{cases}
        \phi(x) & x > 0\\
        0 & x = 0\\
        -\phi(-x) & x < 0
    \end{cases}.\]
    We can now use D'Alembert's Formula
    \[u(x,t) = \frac{1}{2}(\phi_{odd}(x+ct) + \phi_{odd}(x-ct)) + \frac{1}{2c}\int_{x-ct}^{x+ct}\psi_{odd}(s)\,ds.\]
    For $x > ct$, this results in the regular form of D'Alembert's formula:
    \[u(x,t) = \frac{1}{2}(\phi(x+ct) + \phi(x-ct)) + \frac{1}{2c}\int_{x-ct}^{x+ct}\psi(s)\,ds.\]
    Otherwise, we have to modify it per the definition of the odd extension:
    \[u(x,t) = \frac{1}{2}(\phi(x+ct) - \phi(ct-x)) + \frac{1}{2c}\int_{ct-x}^{x+ct}\psi(s)\,ds\]
    For the purposes of this problem, we take $\phi(x) = 1$ and $\psi(x) = 0$. Note that since cosine is an even function, $\cos(x) = \cos(-x)$. We now have our final solution that
    \[\boxed{u(x,t) = \begin{cases}
        1 & x \geq ct\\
        0 & 0 \leq x < ct
    \end{cases}}\]

    \item The Neumann condition indicates that this problem can be solved with an even extension, or
    \[\phi_{even}(x) = \begin{cases}
        \phi(x) & x > 0\\
        0 & x = 0\\
        \phi(-x) & x < 0
    \end{cases}.\]
    The D'Alembert formula is then rewritten as
    \begin{align*}
        u(x,t) &= \frac{1}{2}(\phi_{even}(x+ct) + \phi_{even}(x-ct)) + \frac{1}{2c}\int_{x-ct}^{x+ct}\psi_{even}(s)\,ds\\
        u(x,t) &= \frac{1}{2}(\phi(x+ct) + \phi(ct-x)) + \frac{1}{2c}\left[\int_0^{x+ct}\psi(s)\,ds + \int_0^{ct-x}\psi(s)\,ds\right]
    \end{align*}
    Since $g(x) = 0$ (in this formula, this would be $\phi$), we will need to integrate for this wave equation and its boundaries.\\
    \underline{$x \geq ct$}: Since this section of the first quadrant is not affected, this is just D'Alembert's formula:
    \begin{align*}
        u(x,t) &= \frac{1}{2}(\phi(x+ct) + \phi(x-ct)) + \frac{1}{2c}\int_{x-ct}^{x+ct}\psi(s)\,ds\\
        u(x,t) &= \frac{1}{2c}\int_{x-ct}^{x+ct}ds\\
        u(x,t) &= \frac{1}{2c}s|_{x-ct}^{x+ct}\\
        u(x,t) &= \frac{1}{2c} [x+ct+ct-x]\\
        u(x,t)&= t
    \end{align*}
    \underline{$0 < x < ct$}: We need to use the modified version of D'Alembert's formula.
    \begin{align*}
        u(x,t)&= \frac{1}{2c}\left[\int_0^{x+ct}\psi(s)\,ds + \int_0^{ct-x}\psi(s)\,ds\right]\\
        u(x,t)&= \frac{1}{2c}\left[\int_0^{x+ct}ds + \int_0^{ct-x}ds\right]\\
        u(x,t)&= \frac{1}{2c}\left[s|_0^{x+ct} + s|_0^{ct-x}\right]\\
        u(x,t)&= \frac{1}{2c}\left[x+ct + ct-x\right]\\
        u(x,t)&= t
    \end{align*}
    In other words, whether we have $x \geq ct \text{ or } 0 < x < ct$, the solution is 
    \[\boxed{u(x,t) = t}\]
    
    \item The existence of a Dirichlet boundary condition suggest an odd extension
    \[\phi_{odd}(x) = \begin{cases}
        \phi(x) & x > 0\\
        0 & x = 0\\
        -\phi(-x) & x < 0
    \end{cases}\]
    on D'Alembert's formula,
    \[u(x,t) = \frac{1}{2}(\phi(x+ct) + \phi(x-ct)) + \frac{1}{2c}\int_{x-ct}^{x+ct}\psi(s)\,ds.\]
    A final detail to note is that $\phi(x)$ ($g(x)$ for the sake of the problem) is $0$, so we will need to integrate.
    \underline{$x \geq ct$}: We can just use D'Alembert's formula for this region ($\psi$ written as $h$ to be consistent with the problem, as we have done for the previous problems):
    \begin{align*}
        u(x,t)&= \frac{1}{2c}\int_{x-ct}^{x+ct}h(s)\,ds\\
        u(x,t)&= \frac{1}{2c}\int_{x-ct}^{x+ct}ds\\
        u(x,t)&= \frac{1}{2c}s|_{x-ct}^{x+ct}\\
        u(x,t)&= \frac{1}{2c}[x+ct-x+ct]\\
        u(x,t)&= t
    \end{align*}
    \underline{$0 < x < ct$}: We need to use a form of D'Alembert's formula that uses the odd extension to get our answer:
    \begin{align*}
            u(x,t) &= \frac{1}{2}(\phi_{odd}(x+ct) + \phi_{odd}(x-ct)) + \frac{1}{2c}\int_{x-ct}^{x+ct}\psi_{odd}(s)\,ds\\
            u(x,t) &= \frac{1}{2}(\phi(x+ct) - \phi(ct-x)) + \frac{1}{2c}\int_{ct-x}^{x+ct}\psi(s)\,ds\\
            u(x,t) &= \frac{1}{2c}\int_{ct-x}^{x+ct}ds\\
            u(x,t) &= \frac{1}{2c}s|_{ct-x}^{x+ct}\\
            u(x,t) &= \frac{1}{2c}[x+ct - (ct-x)]\\
            u(x,t) &= \frac{x}{c}
    \end{align*}
    Therefore, the solution to this wave equation problem is:
    \[\boxed{u(x,t) = \begin{cases}
        t & x \geq ct\\
        \frac{x}{c} & 0 \leq x < ct
    \end{cases}}\]
\end{enumerate}
\end{ans}

\end{document}

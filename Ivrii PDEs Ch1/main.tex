\documentclass{article}
\title{Intro to PDEs: Ch 1 HW}
\author{Logan Rhyne, Harley Combest, Roy Galang}
\usepackage[T1]{fontenc}
\usepackage{amsfonts, amsmath, amsthm, amssymb}
\usepackage{mathtools, bigints, empheq}
\usepackage{graphicx, wrapfig, xcolor}
\usepackage{stackrel}
\usepackage[shortlabels]{enumitem}
\usepackage[margin=1.0in]{geometry}
\setlength{\parindent}{0pt}
\theoremstyle{definition}
\newtheorem*{lemma}{Lemma}
\newtheorem*{conj}{Conjecture}
\newtheorem{prob}{}
\newtheorem*{pf}{Proof}
\newtheorem*{dpf}{Disproof}
\renewcommand\qedsymbol{$\blacksquare$}
\renewcommand{\emptyset}{\varnothing}
\renewcommand{\epsilon}{\varepsilon}
\newenvironment{disproof}{\begin{proof}[Disproof]}{\end{proof}}
\newenvironment{ans}{\begin{proof}[Answer]\renewcommand{\qedsymbol}{}}{\end{proof}}
\newenvironment{boldenv}{\bfseries\boldmath}{}

\DeclareMathOperator{\ran}{range}

\begin{document}
	
	\maketitle
	
	\begin{boldenv}
		\underline{Problem 1}. Consider first order equations and determine if they are linear homogeneous, linear inhomogeneous, or nonlinear ($u$
		is an unknown function); for nonlinear equations, indicate if they are also semilinear, or quasilinear:
		\begin{enumerate}[(1), series=problems]
			\item $u_t + xu_x = 0$
			\item $u_t + uu_x = 0$
			\item $u_t + xu_x - u = 0$
			\item $u_t + uu_x + x = 0$
			\item $u_t + u_x - u^2 = 0$
			\item $u_t^2 - u_x^2 - 1 = 0$
			\item $u_x^2 + u_y^2 - 1 = 0$
			\item $xu_x + yu_y + zu_z = 0$
			\item $u_x^2 + u_y^2 + u_z^2 - 1 = 0$
			\item $u_t + u_x^2 + u_y^2 = 0$
		\end{enumerate}
	\end{boldenv}
	
	\begin{ans}
		\begin{enumerate}[(1), series=answers]
			
			\item Let $\mathcal{L}$ be the  operator such that $\mathcal{L}(u)$ represents the left side of the PDE above. Let $c\in\mathbb{R}$. Then
			\begin{align*}
				\mathcal{L}(u+v) &= (u_t + v_t) + x(u_x + v_x)\\
				&= (u_t + xu_x) + (v_t + xv_x)\\
				&= \mathcal{L}(u) + \mathcal{L}(v)\\\\
				\mathcal{L}(cu) &= (cu_t) + x(cu_x)\\
				&=c(u_t + xu_x) = c\mathcal{L}(u).
			\end{align*}
			We have shown that $\mathcal{L}$ is a linear operator, and since there are no functions other than $u$ to be moved to the RHS, this first-order PDE is linear homogeneous.
			
			\item Let $\mathcal{L}$ be the  operator such that $\mathcal{L}(u)$ represents the left side of the PDE above. Let $c\in\mathbb{R}$. Then
			\begin{align*}
				\mathcal{L}(u+v) &= u_t + v_t + (u+v)(u_x + v_x)\\
				&= u_t + v_t + uu_x + uv_x + vu_x + vv_x\\
				&= \mathcal{L}(u) + \mathcal{L}(v) + uv_x + vu_x
			\end{align*}
			Thus, it is nonlinear. In fact, since the coefficient of $u_x$ depends on $u$, this is quasilinear.
			
			\item Since this PDE follows both linearity operations, this PDE is linear. In addition, the function on the right-hand side is equal to 0, therefore, this PDE is linear homogeneous.
			
			\item We only need to use one of the linearity properties to prove that this PDE is nonlinear. Since the coefficient of $u_x$ depends on $u$, this PDE is quasilinear.
			
			\item With one of the linearity properties, this PDE is found to be nonlinear. None of the coefficients of the partial derivatives depend on $u$, therefore, this PDE is semilinear.
			
			\item Not only does this PDE fail the linearity properties test, it also has derivatives that are squared. Therefore, the PDE is fully nonlinear.
			
			\item Similar to the previous PDE, there are squared derivatives. In addition, the PDE does not meet the properties of linearity. This PDE is fully nonlinear.
			
			\item Checking for linearity proves this PDE is linear. Since the right-hand side is equal to 0, the PDE is linear homogeneous.
			
			\item Since the derivatives are squared, this PDE is fully nonlinear.
			
			\item Since there are squared derivatives, this PDE is fully nonlinear.
		\end{enumerate}
		
	\end{ans}
	\begin{boldenv}
		\underline{Problem 2}. Consider equations and determine their order; determine if they are linear homogeneous, linear inhomogeneous or non-linear ($u$ is an unknown function); you need to provide \textit{the most precise} (that means the narrow, but still correct) description:
		\begin{enumerate}[resume*=problems]
			\item $u_t + (1+x^2)u_{xx} = 0$
			\item $u_t - (1 + u^2)u_{xx} = 0$
			\item $u_t + u_{xxx} = 0$
			\item $u_t + uu_x + u_{xxx} = 0$
			\item $u_{tt} + u_{xxxx} = 0$
			\item $u_{tt} + u_{xxxx} + u = 0$
			\item $u_{tt} + u_{xxxx} + \sin{(x)} = 0$
			\item $u_{tt} + u_{xxxx} + \sin{(x)}\sin{(u)} = 0$
		\end{enumerate}
	\end{boldenv}
	
	\begin{ans}
		\begin{enumerate}[resume*=answers]
			\item This PDE follows linearity rules and has a zero on the right-hand side. Therefore, this is a linear homogeneous equation with order 2.
			
			\item This is a second-order PDE that fails the linearity properties, so this is nonlinear. Since there is a coefficient of a derivative that relies on $u$, this PDE is quasilinear.
			
			\item The PDE involves a third derivative, follows linearity rules, and its right hand side is equal to zero. Therefore, this is a third-order linear homogeneous PDE.
			
			\item By the $u_{xxx}$ term, this is a third-order equation. Additionally, since it has a coefficient dependent on $u$ and no nonlinear derivatives, this would be a quasilinear PDE.
			
			\item By the $u_{xxxx}$ term, it is a fourth-order equation. Since the RHS is equal to zero, the coefficients are independent of $u$, and the derivatives are linear (in this case there is no forcing term), this is a linear homogeneous PDE.
			
			\item By the $u_{xxxx}$ term, it is a fourth-order equation. Since the RHS is equal to zero, the coefficients are independent of $u$, and the derivatives and forcing term are linear, this PDE is linear homogeneous.
			
			\item By the $u_{xxxx}$ term, it is a fourth-order equation. Since the the coefficients are independent of $u$ and the derivatives are linear (in this case there is no forcing term), this PDE is linear. However, since $\sin(x)$ is not a coefficient of $u$ or its derivatives, moving it to the RHS shows that this is a linear inhomogeneous PDE.
			
			\item By the $u_{xxxx}$ term, it is a fourth-order equation. The derivatives are all linear, but since moving the $\sin(x)\sin(u)$ term to the RHS yields a function $f(x,u)$ on the RHS, this classifies as semilinear.
		\end{enumerate}
	\end{ans}
	
	\begin{boldenv}
		\underline{Problem 3}. Find the general solutions to the following equations:
		\begin{enumerate}[resume*=problems]
			\item $u_{xy} = 0$
			\item $u_{xy} = 2u_x$
			\item $u_{xy} = e^{x+y}$
			\item $u_{xy} = 2u_x + e^{x+y}$
		\end{enumerate}
		\textit{Hint.} Introduce $v=u_x$ and find it first.
	\end{boldenv}
	
	\begin{ans}
		\begin{enumerate}[resume*=answers]
			\item Integrating with respect to $y$ give us that $u_x = \Phi_1(x)$. Integrating again, this time with respect to $x$ gives us that $\boxed{u(x,y) = \Phi_2(x) + \Phi_3(y)}$ where $\Phi_2(x) = \int \Phi_1(x)dx$.
			
			\item We first substitute $v=u_x$ to get that $v_y = 2v$. We then have the following:
			\begin{align*}
				\frac{\partial v}{\partial y} &= 2v\\
				\int\frac{dv}{v} &= \int2\, d y\\
				\ln{|v|} &= 2y + C(x)\\
				\ln{|u_x|} &= 2y + C(x) \tag{Substituting back in $u_x$ for v}\\
				u_x &= e^{2y}e^{C(x)}\\
				\int u_x\,dx &= \int g_1(x)e^{2y}dx \tag{Letting $g_1(x) = e^{C(x)}$}\\
				\Aboxed{u(x,y) &= g_2(x)e^{2y} + f(y)} \tag{Letting $g_2(x) = \int g_1(x)dx$}
			\end{align*}
			
			\item We begin by separating $e^{x+y}$ into $e^xe^y$. We then integrate both sides and get the following:
			\begin{align*}
				\int u_{xy}dy &= \int e^xe^y dy \\
				u_x &= e^xe^y + f_1(x)\\
				\int u_xdx &= \int\left(e^xe^y + f_1(x)\right)dx\\
				\Aboxed{u(x,y) &= e^xe^y + f_2(x) + g(y)} \tag{Where $f_2(x) = \int f_1(x)dx$}
			\end{align*}
			
			\item First we substitute $v = u_x$ to get $v_y = 2v + e^{x+y}$. It then follows that
			\begin{align*}
				v_y &= 2v + e^{x+y}\\
				v_y - 2v &= e^{x+y}\\
				e^{-2y}v_y - 2e^{-2y}v &= e^{-2y}e^{x+y} = e^{x-y}\\
				\frac{\partial}{\partial y}(ve^{-2y}) &= e^{x-y}\\
				\int \frac{\partial}{\partial y}(ve^{-2y})dy &= \int e^{x-y}dy\\
				ve^{-2y} &= -e^{x-y} + f_1(x)\\
				\frac{ve^{-2y}}{e^{-2y}} &= \frac{-e^{x-y} + f_1(x)}{e^{-2y}}\\
				v &= -e^{x+y} + e^{2y}f_1(x)\\
				u_x &= -e^{x+y} + e^{2y}f_1(x)\\
				\Aboxed{u(x,y) &= -e^{x+y} + e^{2y}f_2(x) + g(y)} \tag{Note that $f_2(x) = \int f_1(x)dx$}
			\end{align*}
		\end{enumerate}
	\end{ans}
	
	\begin{boldenv}
		\underline{Problem 4}. Find the general solutions to the following equations:
		\begin{enumerate}[resume*=problems]
			\item $uu_{xy} = u_xu_y$
			\item $uu_{xy} = 2u_xu_y$
			\item $u_{xy} = u_xu_y$
		\end{enumerate}
		\textit{Hint.} Divide two first equations by $uu_x$ and observe that both the right and left-hand expressions are derivative with respect to $y$ of $\ln(u_x)$ and $\ln(u)$ respectively. Divide the last equation by $u_x$.
	\end{boldenv}
	
	\begin{ans}
		\begin{enumerate}[resume*=answers]
			\item Dividing both sides by $uu_x$ gives us that $\frac{u_{xy}}{u_x} = \frac{u_x}{u}$. We see that equivalently:
			\begin{align*}
				\partial_y(\ln(u_x)) &= \partial_y(\ln(u))\\
				\int \partial_y(\ln(u_x))dy &= \int \partial_y(\ln(u))dy\\
				\ln(u_x) &= \ln(u) + f_1(x)\\
				u_x &= ue^{f_1(x)} = uf_2(x)\\
				\frac{u_x}{u} &= f_2(x)\\
				\partial_x(\ln(u)) &= f_2(x)\\
				\ln(u) &= f_3(x) + g_1(y)\\
				\Aboxed{u(x,y) &= e^{f_3(x)+ g_1(y)} = f_4(x)g_2(y)}
			\end{align*}
			
			\item Again, we divide by $uu_x$, which gives us that $\frac{u_{xy}}{u_x} = \frac{2u_x}{u}$. We then get
			\begin{align*}
				\partial_y(\ln(u_x)) &= \partial_y(2 \ln(u)) = \partial_y(\ln(u^2))\\
				\ln(u_x) &= \ln(u^2) + f_1(x)\\
				e^{\ln(u_x)} &= e^{\ln(u^2) + f_1(x)}\\
				u_x &= u^2f_2(x)\\
				\frac{u_x}{u^2} &= f_2(x)\\
				\partial_x\left(-\frac{1}{u}\right) &= f_2(x)\\
				-\frac{1}{u} &= f_3(x) + g(y)\\
				\Aboxed{u(x,y) &= -(f_3(x) + g(y))^{-1}}
			\end{align*}
			
			\item As the hint gives, we divide by $u_x$ on both sides, getting rid of the $u_x$ on the right-hand side. This allows us to use the same property as the previous problems, yielding:
			\begin{align*}
				\partial_y(\ln(u_x)) &= u_y\\
				\int \partial_y(\ln(u_x)) &= \int u_y\\
				\ln(u_x) &= u + f_1(x)\\
				u_x &= f_2(x)e^u\\
				e^{-u}u_x &= f_2(x)\\
				\frac{\partial}{\partial_x} (e^{-u}) &= f_2(x)\\
				-e^{-u} &= f_3(x) + g_1(y)\\
				-u &= \ln(f_4(x) + g_2(y))\\
				\Aboxed{u &= -\ln(f_4(x) + g_2(y))}
			\end{align*}
			
		\end{enumerate}
	\end{ans}
	
	\begin{boldenv}
		\underline{Problem 5}. Find general solutions to the following \textit{linear homogeneous equations}:
		\begin{enumerate}[resume*=problems]
			\item $u_{xxy} = 0$
			\item $u_{xxyy} = 0$
			\item $u_{xxxy} = 0$
			\item $u_{xyz} = 0$
			\item $u_{xyzz} = 0$
			\item $u_{xxy} = \sin(x)\sin(y)$
			\item $u_{xxy} = \sin(x) + \sin(y)$
			\item $u_{xxyy} = \sin(x)\sin(y)$
			\item $u_{xxyy} = \sin(x) + \sin(y)$
			\item $u_{xxxy} = \sin(x)\sin(y)$
			\item $u_{xxxy} = \sin(x) + \sin(y)$
			\item $u_{xyz} = \sin(x)\sin(y)\sin(z)$
			\item $u_{xyz} = \sin(x) + \sin(y) + \sin(z)$
			\item $u_{xyz} = \sin(x) + \sin(y)\sin(z)$
		\end{enumerate}
	\end{boldenv}
	
	\begin{ans}
		\begin{enumerate}[resume*=answers]
			\item \begin{align*}
				u_{\mathrm{xxy}} &= 0\\
				u_{\mathrm{xy}} &= \Phi_1(y)\\
				u_y &= \Phi_1(y)x + \psi(y) \\
				\Aboxed{u(x,y) &= \Phi_2(y)x + \psi(y) + \varphi(x)}
			\end{align*}
			
			
			\item \begin{align*}
				u_{\mathrm{xxyy}} &= 0\\
				u_{\mathrm{xxy}} &= \varphi_1(x) \\
				u_{\mathrm{xx}} &= y\varphi_1(x) + \beta_1(x) \\
				u_x &= y\varphi_2(x) + \beta_2(x) + \gamma_1(y) \\
				\Aboxed{u(x, y) &= y\varphi_3(x) + \beta_3(x) + x\gamma_1(y) + \rho_1(y)}
			\end{align*}
				
				\item \begin{align*}
				u_{\mathrm{xxxy}} &= 0 \\
				u_{\mathrm{xxy}} &= \varphi_1(y)\\ 
				u_{\mathrm{xy}} &= \varphi_1(y)x + \beta_1(y) \\
				u_y &= \frac{1}{2}\varphi_1(x)x^2 + \beta_1(y)x + \gamma_1(y) 	
				\\ \Aboxed{u(x, y) &= \frac{1}{2}x^2\varphi_2(x) + x\beta_2(x) + x\gamma_2(y) + \rho_1(x)}
			\end{align*}
				
				\item \begin{align*}
					u_{\mathrm{xyz}} &= 0 
					\\ u_{\mathrm{xy}} &= \varphi_1(x, y) 
					\\ u_{\mathrm{x}} &= \varphi_2(x, y) + \beta_1(x, z)	
					\\ \Aboxed{u(x, y) &= \varphi_3(x, y) + \beta_2(x) + \gamma_1(y,z)}
				\end{align*}
				
				\item \begin{align*}
					u_{\mathrm{xyzz}} &= 0 
					\\ u_{\mathrm{xyz}} &= \varphi_1(x, y) 
					\\ u_{\mathrm{xy}} &= z\varphi_1(x, y) + \beta_1(x, y)	
					\\ u_x &= z\varphi_2(x, y) + \beta_2(x, y)	+ \gamma_1(x, z)
					\\ \Aboxed{u(x, y) &= z\varphi_3(x, y) + \beta_3(x, y) + \gamma_2(x,z) + \rho_1(y, z)}
				\end{align*}
				
				\item \begin{align*}
					u_{\mathrm{xxy}} &= \sin(x)\sin(y) 
					\\ u_{\mathrm{xx}} &= -\sin(x)\cos(y) + \alpha_1(x)
					\\ u_x &= \cos(x)\cos(y) + \alpha_2(x) + \beta_1(y)
					\\ \Aboxed{u(x,y) &= \sin(x)\cos(y) + \alpha_3(x) + x\beta(y) + \gamma(y)}
				\end{align*}
				
				\item \begin{align*}
					u_{\mathrm{xxy}} &= \sin(x)+\sin(y) 
					\\ u_{\mathrm{xx}} &= y\sin(x) - \cos(y) + \alpha_1(x)
					\\ u_x &= -y\cos(x) - x\cos(y) + \alpha_2(x) + \beta_1(y)
					\\ \Aboxed{u(x,y) &= -y\sin(x) - \frac{1}{2}x^2\cos(y) + \alpha_3(x) + x\beta(y) + \gamma(y)}
				\end{align*}
				
				\item \begin{align*}
					u_{\mathrm{xxyy}} &= \sin(x)\sin(y) 
					\\ u_{\mathrm{xxy}} &= -\sin(x)\cos(y) + \alpha_1(x)
					\\ u_{\mathrm{xx}} &= -\sin(x)\sin(y) + y\alpha_1(x) + \beta_1(x)
					\\ u_x &= \cos(x)\sin(y) + y\alpha_2(x) + \beta_2(x) + \gamma_1(y)
					\\ \Aboxed{u(x,y) &= \sin(x)\sin(y) + y\alpha_3(x) + \beta_3(x) + x\gamma_1(y) + \delta_1(y)}
				\end{align*}
				
				\item \begin{align*}
					u_{\mathrm{xxyy}} &= \sin(x)+\sin(y) 
					\\ u_{\mathrm{xxy}} &= y\sin(x) - \cos(y) + \alpha_1(x)
					\\ u_{\mathrm{xx}} &= \frac{1}{2}y^2\sin(x) - \sin(y) + y\alpha_1(x) + \beta_1(x)
					\\ u_x &= -\frac{1}{2}y^2\cos(x) - x\sin(y) + y\alpha_2(x) + \beta_2(x) + \gamma_1(y)
					\\ \Aboxed{u(x,y) &= -\frac{1}{2}y^2\sin(x) - \frac{1}{2}x^2\sin(y) + y\alpha_3(x) + \beta_3(x) + x\gamma_1(y) +\delta_1(y)}
				\end{align*}
				
				\item \begin{align*}
					u_{\mathrm{xxxy}} &= \sin(x)\sin(y) 
					\\ u_{\mathrm{xxx}} &= -\sin(x)\cos(y) + \alpha_1(x)
					\\ u_{\mathrm{xx}} &= \cos(x)\cos(y) + \alpha_2(x) + \beta_1(y)
					\\ u_x &= \sin(x)\cos(y) + \alpha_3(x) + x\beta_1(y) + \gamma_1(y)
					\\ \Aboxed{u(x,y) &= -\cos(x)\cos(y) + \alpha_4(x) + \frac{1}{2}x^2\beta_1(y) + x\gamma_1(y) + \delta_1(y)}
				\end{align*}
				
				\item \begin{align*}
					u_{\mathrm{xxxy}} &= \sin(x)+\sin(y) 
					\\ u_{\mathrm{xxx}} &= y\sin(x) - \cos(y) + \alpha_1(x)
					\\ u_{\mathrm{xx}} &= -y\cos(x) - x\cos(y) + \alpha_2(x) + \beta_1(y)
					\\ u_x &= -y\sin(x) - \frac{1}{2}x^2\cos(y) + \alpha_3(x) + x\beta_1(y) + \gamma_1(y)
					\\ \Aboxed{u(x,y) &= y\cos(x) - \frac{1}{3}x^3\sin(y) + y\alpha_4(x) + \frac{1}{2}x^2\beta_1(y) + x\gamma_1(y) +\delta_1(y)}
				\end{align*}
				
				\item \begin{align*}
					u_{\mathrm{xyz}} &= \sin(x)\sin(y)\sin(z)
					\\ u_{\mathrm{xy}} &= \sin(x)\sin(y)\cos(z) + \alpha_1(x,y)
					\\ u_x &= \sin(x)\cos(y)\cos(z) + \alpha_2(x, y) + \beta_1(x, z)
					\\ \Aboxed{u(x,y, z) &= -\cos(x)\cos(y)\cos(z) + z\alpha_2(x,y) + \beta_2(x, z) + \gamma_1(y, z)}
				\end{align*}
				
				\item \begin{align*}
					u_{\mathrm{xyz}} &= \sin(x)+\sin(y)+\sin(z)
					\\ u_{\mathrm{xy}} &= z\sin(x) + z\cos(y) - \cos(z) + \alpha_1(x, y)
					\\ u_x &=  yz\sin(x) - z\cos(y) - y\cos(z) + \alpha_2(x) + \beta_1(x,z)
					\\ \Aboxed{u(x,y) &= -yz\cos(x) - xz\cos(y) + xy\cos(z) + \alpha_3(x, y) + \beta_2(x,z) +\delta_1(y,z)}
				\end{align*}
				
				\item \begin{align*}
					u_{\mathrm{xyz}} &= \sin(x)-\sin(y)\sin(z)
					\\ u_{\mathrm{xy}} &= z\sin(x) - \sin(y)\cos(z) + \alpha_1(x, y)
					\\ u_x &=  yz\sin(x) + \cos(y)\cos(z) + \alpha_2(x, y) + \beta_1(x,z)
					\\ \Aboxed{u(x,y) &= -yz\cos(x) + x\cos(y)\cos(z) + \alpha_3(x, y) + \beta_2(x,z) +\delta_1(y,z)}
				\end{align*}
			\end{enumerate}
		\end{ans}
		
	\end{document} 
